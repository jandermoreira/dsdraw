%%
%% DSDRAW: file
%% This file is loaded by ../dsdraw.sty
%%
%
\ds@requirelibrary{record}


%%%%%%%%%%%%%%%%%%%%%%%%%%%%%%%%%%%%%%%%%%%%%%%%%%%%%%%%%%%%%%%%
% Options
%

\pgfkeys{
	/dsdraw/file/.cd,
	%
	show frame/.is if = ds@file@showframe,
	hide frame/.code = {\ds@file@showframefalse},
	%
	name/.store in = \ds@file@chainname,
	record height/.store in = \ds@file@recordheight,
	show offsets/.is if = ds@file@showoffsets,
	hide offsets/.code = {\ds@file@showoffsetsfalse},
	%
	kbyte width/.store in = \ds@file@kbytewidth,
	block size/.store in = \ds@file@blocksize,
	block size = 1024,
	%
	records/.store in = \ds@file@recordsinblock,
	%
	record options/.code = {\dsrecordset{#1}},
	field options/.code = {\dsfieldset{#1}},
	tikz options/.code = {\dstikzset{#1}},
}


%%%%%%%%%%%%%%%%%%%%%%%%%%%%%%%%%%%%%%%%%%%%%%%%%%%%%%%%%%%%%%%%
% Variables
%

\newcounter{dsfrecordnumber} % current number of records in file
\newcounter{dsfbytenumber} % current number of bytes in file
\newcounter{dsfblocknumber} % current block
\newcounter{dsfrecordnumberinblock} % number of records in block

\newif\ifds@file@firstblockinfile
\newif\ifds@file@showframe
\newif\ifds@file@showoffsets

%%%%%%%%%%%%%%%%%%%%%%%%%%%%%%%%%%%%%%%%%%%%%%%%%%%%%%%%%%%%%%%%
% Utilities
%

% dsfileset: shortcut to file-related pgfkeys
\newcommand{\dsfileset}[1]{
%	\def\keysmacro{/dsdraw/file/.cd, #1}
%	\edef\ekeysmacro{\keysmacro}%
%	\pgfkeysalsofrom{\ekeysmacro}%
	\pgfkeys{/dsdraw/file/.cd, #1}
}

%%%%%%%%%%%%%%%%%%%%%%%%%%%%%%%%%%%%%%%%%%%%%%%%%%%%%%%%%%%%%%%%
% Defaults
%

\dsfileset{%
	name = file,
	hide frame,
	records = 0, % i.e., any
	record height = 1em,
	hide offsets,
	kbyte width = 10em,
}


%%%%%%%%%%%%%%%%%%%%%%%%%%%%%%%%%%%%%%%%%%%%%%%%%%%%%%%%%%%%%%%%
% Drawing stuff
%

\tikzset{
	file/.style = {
		inner xsep = 0.1em,
		inner ysep = 0.1em,
		fit={
			(\ds@file@chainname-record-0.north west)
			(current bounding box.south east)
		}
	},
	block/.style = {
		draw,
		inner xsep = 0.3em,
		inner ysep = 0.2em,
		fit = {
			(\ds@file@chainname-record-\@firstrecordinblock.north west)
%			(current bounding box.south east)
			(\ds@file@chainname-record-\@lastrecordinblock.south east)
		},
	},
	file offset/.style = {
		transform shape,
		scale = 0.4,
		text depth = 0.25ex,
		inner sep = 1ex,
		outer sep = 0pt,
	},
	file record v/.style = {
%		draw = yellow, dashed, very thick, text width = 2em,
		minimum height = \ds@file@recordheight,
		inner sep = 0pt,
		outer sep = 0pt,
		anchor = north west,
		on chain,
	},
	file record h/.style = {
		draw,
		fill = white,
		text height = \ds@file@recordheight,
		text depth = \dimexpr \ds@file@recordheight/3,
		align = center,
		inner sep = 0em,
		outer sep = 0em,
		anchor = west,
		on chain,
	},
}

\ExplSyntaxOn
% ds_call_dsnamedrecord:nnn: auxiliary function to call \dsnamedrecord
%	with all parameters
%	#1: options for \dsrecordset
%	#2: record name (see \dsnewnamedrecord)
%	#3: comma-separated list of fields contents
\cs_new_protected:Npn \ds_call_dsnamedrecord:nnn #1#2#3 {
	\begingroup
	\edef\@name{\ds@file@chainname-record-\thedsfrecordnumber}
	\dsrecordset{
		name = \@name,
		#1,
	}
	\xdef\ds@file@currentrecordname{\ds@record@chainname} % hack!
	\dsnamedrecord[#1]{#2}{#3}
	\endgroup
}

% ds@file@drawrecordhelp: process a single record (called from
%	\ds@file@drwnamedrecord) and check if there are optional configuration
%	#1: record name (see \dsnewnamedrecord)
%	#2: the record: comma-separated list of fields + (optional) "/" list
%		of comma-separated list of record options
\NewExpandableDocumentCommand{\ds@file@drawrecordhelper} { mm } {
	\seq_set_split:Nnn \l_fields_options_seq { / } { #2 }
	\tl_set:Nn \l_fields_tl { \seq_item:Nn \l_fields_options_seq { 1 } }
	\tl_set:Nn \l_options_tl { \seq_item:Nn \l_fields_options_seq { 2 } }
	\exp_args:Neee \ds_call_dsnamedrecord:nnn
		{ \l_options_tl } { #1 } { \l_fields_tl }
}
\ExplSyntaxOff

% ds@file@drawnamedrecord: handles positioning and scope for each record
%	#1: record name (see \dsnewnamedrecord)
%	#2: the record: comma-separated list of fields + (optional) "/" list
%		of comma-separated list of record options
\NewDocumentCommand{\ds@file@drawnamedrecord} {mm} {
	\edef\@currentrecord{\ds@file@chainname-record-\thedsfrecordnumber}
	\node[file record v] (\@currentrecord) {}; % reference node
	\begin{scope}[shift = (\@currentrecord.north west)]
		\ds@file@drawrecordhelper{#1}{#2}
		\ifds@file@showoffsets
			\node[anchor = east, font = \ttfamily, scale = 0.7]
				at ($(\@currentrecord.west) - (1em, 0)$) {\nhex{8}{\thedsfbytenumber}};
		\fi
		\addtocounter{dsfbytenumber}{\thedsrbytenumber}
		\pgfnodealias{\@currentrecord}{\ds@file@currentrecordname}
		\ifnum\thedsrbytenumber>\@longestrecord\relax
			\global\edef\@longestrecord{\thedsrbytenumber}
		\fi
	\end{scope}
	\stepcounter{dsfrecordnumber}
	\stepcounter{dsfrecordnumberinblock}
}


\ExplSyntaxOn
% \ds@file@drawrecord: helper to draw a single record in a structure file
%	#1: record label / length in bytes
\NewDocumentCommand{\ds@file@drawrecord} {m} {
	\seq_set_split:Nnn \l_record_seq { / } { #1 }
	\tl_set:Nn \l_label_tl { \seq_item:Nn \l_record_seq { 1 } }
	\tl_set:Nn \l_length_tl { \seq_item:Nn \l_record_seq { 2 } }
	\ds@file@calldrawrecord { \l_label_tl } { \l_length_tl }
}

\NewDocumentCommand{\ds@file@drawnamedrecordexpand} {mm} {
	\exp_args:Noo \ds@file@drawnamedrecord { #1 } { #2 }
}

% dsfilenamedhelper: auxiliary to loop through a list of records
%	#1: record name (see \dsnewnamedrecord)
%	#2: a comma-separated list of records (fields contents/options)
\NewDocumentCommand{\dsfilenamedhelper}{mm}{
	\clist_map_inline:nn {#2} { \ds@file@drawnamedrecord {#1} {##1} }
}
\ExplSyntaxOff

% ds@file@calldrawrecord: draw a single record
%	#1: record label
%	#2: record length in bytes
\NewDocumentCommand{\ds@file@drawstructurerecord} {mm} {
	\edef\@currentnode{\ds@file@chainname-record-\thedsfrecordnumber}
	\pgfmathparse{\ds@file@kbytewidth/\ds@file@blocksize * #2 - 0.1em}
	\edef\@nodewidth{\pgfmathresult pt}
	
	% draw record
	\node[file record h, text width = \@nodewidth]
		(\@currentnode) {#1};
	
	% draw offsets
	\node[file offset, anchor = north west]
		at (\@currentnode.south west) {\thedsfbytenumber};
	\node[file offset, anchor = north east]
		at (\@currentnode.south west) {\thedsfbytenumber};
	\addtocounter{dsfbytenumber}{#2}
}

\NewDocumentCommand{\ds@file@drawstructureblocks} {} {
	\pgfmathsetmacro\@remainingbytes{int(mod(\thedsfbytenumber, \ds@file@blocksize))}
	\ifnum\@remainingbytes=0\relax
		\pgfmathsetmacro\@lastblock{int(\thedsfbytenumber / \ds@file@blocksize) - 1}
	\else
		\pgfmathsetmacro\@lastblock{int(\thedsfbytenumber / \ds@file@blocksize)}
	\fi
	\begin{scope}[on background layer]
		\foreach \n in {0, ..., \@lastblock}{
			\draw [draw, xshift = -0.05em]
				(\n * \ds@file@kbytewidth, 1em)
				rectangle
				(\n * \ds@file@kbytewidth + \ds@file@kbytewidth, -1em);
		}
	\end{scope}
}

\ExplSyntaxOn
% ds@file@filestructurehelper: auxiliary to loop through a list of
%	records and sizes
%	#1: comma-separated list of pairs record-label/size-in-bytes
\NewDocumentCommand{\ds@file@filestructurehelper} {m} {
	\clist_map_inline:nn {#1} {
		% get label and size
		\seq_set_split:Nnn \l_record_seq { / } { ##1 }
		\tl_set:Nn \l_label_tl { \seq_item:Nn \l_record_seq { 1 } }
		\tl_set:Nn \l_size_tl { \seq_item:Nn \l_record_seq { 2 } }
		
		% draw
		\ds@file@drawstructurerecord { \l_label_tl } { \l_size_tl }
	}
	\ds@file@drawstructureblocks
}
\ExplSyntaxOff



%%%%%%%%%%%%%%%%%%%%%%%%%%%%%%%%%%%%%%%%%%%%%%%%%%%%%%%%%%%%%%%%
% Interface

% dsfilerecords: draws a list of records
\NewExpandableDocumentCommand{\dsfilerecords} {mm} {
	\dsfilenamedhelper{#1}{#2}
}

\NewDocumentCommand{\dsfilenamedrecord} {smm} {
	\IfBooleanTF{#1}{
		\ds@file@drawnamedrecordexpand{#2}{#3} % fully expand arguments
	}{
		\ds@file@drawnamedrecord{#2}{#3} % regular tokens
	}
}

\NewDocumentEnvironment{dsfile}{O{}}{
	\begingroup
	\dsfieldset{%
		terminator byte = {},
		fill byte = {},
	}
	\tikzset{control bytes/.append style = {fill = none}}
	\begin{dsfile*}[#1]
}{
	\end{dsfile*}
	\endgroup
}

\NewDocumentEnvironment{dsfile*}{O{}}{
	\begingroup
	
	% initializations and defaults
	\setcounter{dsfrecordnumber}{0}
	\setcounter{dsfbytenumber}{0}
	\setcounter{dsfblocknumber}{0}
	
	\dsfileset{record height = 1.5em}
	\dsrecordset{%
		hide record frame,
		compact field height = \ds@file@recordheight,
	}	
	
	% code
	\dsfileset{#1}
	\ifds@record@recordframe
		\def\@dist{0.3em}
	\else
		\def\@dist{0.1em}
	\fi
	\begin{scope}
	\dstikzset{%
		start chain = \ds@file@chainname{} going below,
		node distance = \@dist,
	}
	\ds@file@firstblockinfiletrue
	\def\@longestrecord{0}
}{
	\begingroup % to frame or not to frame
	\ifds@file@showframe
		\tikzset{file/.append style = {draw}}
	\fi
	\node[file] (\ds@file@chainname) {};
	\endgroup
	\end{scope}
	\endgroup
}

\NewDocumentEnvironment{dsblock} {m} {
	\ifds@file@firstblockinfile
		\global\ds@file@firstblockinfilefalse
	\else
		% give some space between blocks
		\node[on chain,
			text height = \dimexpr 2pt + 0.2em \relax, % height of the node
			outer sep = -0.8pt, % reduce the node to 0pt
			inner sep = 0pt] {};
	\fi
	\edef\@firstrecordinblock{\thedsfrecordnumber}
	\setcounter{dsfrecordnumberinblock}{0}
}{
	\pgfmathsetmacro\@remain{%
		int(\ds@file@recordsinblock - \thedsfrecordnumberinblock)}
%	\dsnewnamedrecord{__empty_record}{\dsfieldfixed*{\@longestrecord}}
	\@whilenum\thedsfrecordnumberinblock<\ds@file@recordsinblock\relax\do{
		\dsfilenamedrecord{#1}{{},{},{},{},{},{},{},{},{},{},{},{}}%/hide field frame}
	}
	\pgfmathsetmacro\@lastrecordinblock{int(\thedsfrecordnumber - 1)}
	\node[block] (block) {};
	\node[below right, font = \itshape, scale = 0.75]
		at (block.north east)
		{bloco $\thedsfblocknumber$};
	\stepcounter{dsfblocknumber}
}


% dsfilestructure: file with records and block boundaries (horizontal)
%	#1: Options (optional)
%	#2: comma-separated list of records and sizes
\NewDocumentCommand{\dsfilestructure} { O{}m } {
	\begingroup
	\dsfileset{#1}
	\def\@dist{0.1em}
	\setcounter{dsfbytenumber}{0}
	\setcounter{dsfrecordnumber}{0}
	\begin{scope}
		\dstikzset{%
			start chain = \ds@file@chainname{} going right,
			node distance = \@dist,
		}
		\ds@file@filestructurehelper{#2}
	\end{scope}
	\endgroup
}


\NewDocumentCommand{\dsfileskip} { O{1}m } {
%	\node...
}


%% EOF