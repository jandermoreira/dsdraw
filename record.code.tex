%
% DSDRAW: record
% This file is loaded by ../dsdraw.sty
%
\input{binhex}

%%%%%%%%%%%%%%%%%%%%%%%%%%%%%%%%%%%%%%%%%%%%%%%%%%%%%%%%%%%%%%%%
% Parameters
%
\pgfkeys{
	/dsdraw/record/.cd,
	% todo: maybe use a name
	name/.store in = \ds@record@chainname,
	%
	scale/.store in = \ds@record@lengthscale,
	field font/.store in = \ds@record@fieldfont,
	field font = \fontfamily{cmtt}\selectfont,
	field font color/.store in = \ds@record@fieldfontcolor,
	field height/.store in = \ds@record@fieldheight,
	compact field height/.store in = \ds@record@byteheight,
	%
	record frame color/.store in = \@ds@record@recordframecolor,
	hide record frame/.code = {\ds@record@recordframefalse},
	show record frame/.code = {\ds@record@recordframetrue},
	%
	field frame color/.store in = \ds@record@recordfieldcolor,
	hide field frame/.code = {\ds@record@fieldframefalse},
	show field frame/.code = {\ds@record@fieldframetrue},
	byte length/.store in = \ds@record@bytelength,
	length/.store in = \ds@record@recordbytelength, % in bytes
	%
	terminator/.is if = ds@record@useterminator,
	prefix/.is if = ds@record@useprefix,
	deleted/.is if = ds@record@usedeletion,
	next/.store in = \ds@record@nextrecord,
	%
	byte offset/.store in = \ds@record@firstbyteoffset,
	show structure offsets/.code = {\ds@record@showstructureoffsettrue},
	hide structure offsets/.code = {\ds@record@showstructureoffsetfalse},
	%
	latin/.is if = ds@record@uselatin,
	%
	tikz/.code = {\dstikzset{#1}},
	%
	field options/.code = {\dsfieldset{#1}},
	tikz options/.code = {\dstikzset{#1}},
}

\pgfkeys{
	/dsdraw/field/.cd,
	%
	terminator byte/.store in = \ds@record@fieldterminator,
	fill byte/.store in = \ds@record@fieldfill,
	%
	tikz options/.code = {\dstikzset{#1}},
}


%%%%%%%%%%%%%%%%%%%%%%%%%%%%%%%%%%%%%%%%%%%%%%%%%%%%%%%%%%%%%%%%
% Utilities
%

% dsrecordset: shortcut to set record-related pgfkeys
\newcommand{\dsrecordset}[1]{%
	\pgfkeys{/dsdraw/record/.cd, #1}%
}

% dsrecordset: shortcut to set record-related pgfkeys
\newcommand{\dsfieldset}[1]{%
	\pgfkeys{/dsdraw/field/.cd, #1}%
}


%%%%%%%%%%%%%%%%%%%%%%%%%%%%%%%%%%%%%%%%%%%%%%%%%%%%%%%%%%%%%%%%
% Colors
%
%\colorlet{fieldfontcolor}{.}
%\colorlet{fieldfontcolor}{.}
%\colorlet{fieldframecolor}{.}
%\colorlet{controlbytecolor}{.}

%%%%%%%%%%%%%%%%%%%%%%%%%%%%%%%%%%%%%%%%%%%%%%%%%%%%%%%%%%%%%%%%
% Variables
%

\def\ds@record@recordterminator{\dsbyte{01}}  % ends a record
\def\recordfillbyte{\dsbyte{FE}}  % unused bytes in a record
\def\nulladdress{0xFFFF}  % NULL address

\newif\ifds@record@recordframe
\newif\ifds@record@fieldframe

\newif\ifds@record@useprefix
\newif\ifds@record@useterminator
\newif\ifds@record@usedeletion
\newif\ifds@record@uselatin
\newif\ifds@record@showstructureoffset

%% defaults

\def\ds@record@nextrecord{\nulladdress}
\def\ds@record@chainname{record}
\def\ds@record@firstbyteoffset{0}
\def\ds@record@bytelength{1.2ex}
\def\ds@record@lengthscale{1}
\dsrecordset{%
	field font color = .,
	field height = 1.8em,
	compact field height = 1.4em,
	show record frame,
	show structure offsets,
	show field frame,
	length = 0,
	next = hide, % prevents showing the field 'next'
}

\dsfieldset{%
	terminator byte = {\dsbyte{00}},
	fill byte = {\dsbyte{FF}}
}

%%%%%%%%%%%%%%%%%%%%%%%%%%%%%%%%%%%%%%%%%%%%%%%%%%%%%%%%%%%%%%%%
% Drawing stuff
%
\usetikzlibrary{chains, calc, shapes, shapes.multipart, fit, backgrounds}

\newcounter{dsrrecordoffset} % byte offset in a record
\newcounter{dsrfieldspaces} % number of spaces between fields in a record
\newcounter{dsrbytenumber} % number of the byte in the record
\newcounter{dsrfieldnumber} % number of the field in the record
\newcounter{dsrdatabyte} % number of the current data byte

\def\ds@record@fieldwidth{2em}
\def\ds@record@bytewidth{0.6em}

\tikzset{
	byte/.style = {
		align = center,
		font = \ds@record@fieldfont,
		text = \ds@record@fieldfontcolor,
		text width = \ds@record@fieldwidth,
		text depth = \dimexpr \ds@record@fieldwidth/4 \relax, % 0.25
		inner sep = 0pt,
		outer sep = 0pt,
		text height = \dimexpr \ds@record@fieldwidth/2 \relax, % 0.5
		minimum height = \ds@record@fieldheight,
		anchor = north west,
		draw,
	},
	compact byte/.style = {
		text width = \ds@record@bytewidth,
		text depth = \dimexpr \ds@record@byteheight/10 \relax,
		text height = \dimexpr \ds@record@byteheight/2 \relax, 
		minimum height = \ds@record@byteheight,
		anchor = north west,
		draw = none,
	},
	field space/.style = {
		text width = 0.25ex,
		text height = \dimexpr \ds@record@byteheight \relax, 
		inner sep = 0pt,
		outer sep = 0pt,
		on chain,
		anchor = north west,
		draw = none,
	},
	record text/.style = {
		font = \ds@record@fieldfont,
		text = \ds@record@fieldfontcolor,
		minimum height = \ds@record@fieldheight,
		text depth = 0.6ex,
		inner sep = 0pt,
		outer sep = 0pt,
	},
	record/.style = {
		draw,
		anchor = north west,
		on chain,
		record text,
	},
	record offset/.style = {
		transform shape,
		scale = 0.4,
		text depth = 0.25ex,
		inner sep = 1ex,
		outer sep = 0pt,
	},
	control bytes/.style = {
		fill = .!10,
	},
	control bytes block/.style = {
		draw = none,
	},
	deleted bytes/.style = {
		fill = black,
		text = white,
	},
	deleted bytes block/.append style = {
		draw = white,
	},
}

\ExplSyntaxOn
\cs_new_protected:Nn \ds_process_token:n {
	\bool_lazy_or:nnTF {! \tl_if_single_p:n {#1}} {\token_if_cs_p:N #1}{
		#1
	}{
		\ds_process_highbyte:en { `\token_to_str:N #1 } { #1 }
	}
}

\cs_new:Nn \ds_process_highbyte:nn {
	\int_compare:nTF { #1>127 } { \dsbyte{\int_to_Hex:n { #1 }} } { #2 }
}

\cs_generate_variant:Nn \ds_process_highbyte:nn { en }

\NewDocumentCommand{\ds@record@utfviii}{m}{
	\ds_process_token:n {#1}
}
\ExplSyntaxOff

% #1: (star) the byte is not data and therefore, not counted
\NewDocumentCommand{\ds@record@drawbyte} {sm} {
	\IfBooleanTF{#1}{
		\node[byte, on chain]
			(\ds@record@chainname-byte-\thedsrbytenumber)
			{\ds@record@utfviii{#2}};
	}{
		\pgfmathparse{int(\thedsrdatabyte + 1)}
		\ifnum\pgfmathresult>\ds@record@skipbytes\relax
			\node[byte, on chain]
				(\ds@record@chainname-byte-\thedsrbytenumber)
				{\ds@record@utfviii{#2}};
		\fi
		\stepcounter{dsrdatabyte}
	}
	% todo: print byte offsets?
%	\node[scale=0.3, anchor = north, outer sep = 2pt]
%		at (\ds@record@chainname-byte-\thedsrbytenumber.south)
%		{\thedsrbytenumber};
	\stepcounter{dsrbytenumber}
}

\NewDocumentCommand{\ds@record@drawbytelatin} {sm} {
	\IfBooleanTF{#1}{
		\node[byte, on chain] {#2};
	}{
		\node[byte, on chain]
			(\ds@record@chainname-byte-\thedsrbytenumber) {#2};
		\stepcounter{dsrdatabyte}
	}
}

\ExplSyntaxOn
\NewDocumentCommand{\ds@record@drawfield}{sm}{
	\ifds@record@uselatin
		\str_set_convert:Nnnn \l_latin_str {jander} {utf8} {latin1}
		\str_map_function:NN \l_latin_str \ds@record@drawbytelatin
	\else
		\tl_map_function:nN {#2} \ds@record@drawbyte
	\fi
	\ifds@record@fieldframe
		\node[draw,
			inner~sep = 0pt,
			fit = {
				(\ds@firstfieldbyte.north~west)
				(\tikzchaincurrent.south~east)
			}] (\ds@record@chainname-field-\thedsrfieldnumber) {};
	\fi
	\IfBooleanF{#1}{\stepcounter{dsrfieldnumber}}
}

\NewExpandableDocumentCommand{\ds@record@bytecount}{m} {
	\tl_count:n {#1}
}
\ExplSyntaxOff


%%%%%%%%%%%%%%%%%%%%%%%%%%%%%%%%%%%%%%%%%%%%%%%%%%%%%%%%%%%%%%%%
% Commands
%

% a byte in format 0x... (horizontal or vertical, according to
% ds@bytehorizontal)
\newif\ifds@bytehorizontal
\ds@bytehorizontaltrue
\newlength{\@bytelength}
\newcommand{\dsbyte}[1]{%
	\ifds@bytehorizontal%
		\raisebox{0em}{\scalebox{0.65}[0.8]{\texttt{0x#1}}}%
	\else%
		\settowidth{\@bytelength}{%
			\pgfinterruptpicture\texttt{0x#1}\endpgfinterruptpicture%
		}%
		\pgfmathsetmacro\@scale{0.75 * \ds@record@byteheight/\@bytelength}%
		\raisebox{-0.2em}{\rotatebox{90}{\scalebox{\@scale}{\texttt{0x#1}}}}%
	\fi%
}


% ds@record@deletedfront: code at front of record when deleted
\newcommand{\ds@record@deletedfront}{
	\ifds@record@usedeletion
		\begingroup
		\dstikzset{byte/.append style = {compact byte, deleted bytes}}
		\ds@bytehorizontalfalse
		
		% record length: bytes 0 and 1
		\ds@record@drawbyte*{}
		\ds@record@drawbyte*{}

		% deletion magick number: bytes 2 and 3
		\ds@record@drawbyte*{\dsbyte{FF}}
		\ds@record@drawbyte*{\dsbyte{FF}}
		\node [deleted bytes block, 
			fit = {
				(\ds@record@chainname-byte-2)
				(\ds@record@chainname-byte-3)
			},
			inner sep = 0pt] {};

		% link to next record: bytes 4 to 7
		\def\@hide{hide}
		\ifx\@hide\ds@record@nextrecord\relax
		\else
			\dshexset{link}[big]{\ds@record@nextrecord}[4]
			\foreach \b [count=\n from 0] in {1, ..., \dshexcount{link}} {
				\ds@record@drawbyte*{\dsbyte{\dshexbyte{link}{\n}}}
			}
			\node [deleted bytes block, 
				fit = {
					(\ds@record@chainname-byte-4)
					(\ds@record@chainname-byte-7)
				},
				inner sep = 0pt] {};
		\fi
		\xdef\ds@firstfieldbyte{\ds@record@chainname-byte-\thedsrbytenumber}
		\xdef\ds@record@skipbytes{\thedsrbytenumber}
		\setcounter{dsrbytenumber}{0} % reset counter for data bytes
		\endgroup
		\dsrecordset{%
			field font color = black!40,
			hide field frame, % garbage is no longer fields
		}
		\tikzset{control bytes/.append style = {fill = none}}
	\fi
}

% ds@record@deletedback: code for deteled records at the end
\newcommand{\ds@record@deletedback}{
	\ifds@record@usedeletion
		\begingroup
		\tikzset{byte/.append style = {compact byte}}
		\ds@bytehorizontalfalse
		\dshexset{size}[big]{\thedsrbytenumber}
		\node[byte, deleted bytes] at (\ds@record@chainname-byte-0.north west)
			{{\dsbyte{\dshexbyte{size}{0}}}};
		\node[byte, text = white] at (\ds@record@chainname-byte-1.north west)
			{{\dsbyte{\dshexbyte{size}{1}}}};
		\node [deleted bytes block,
			fit = {
				(\ds@record@chainname-byte-0)
				(\ds@record@chainname-byte-1)
			},
			inner sep = 0pt] {};
		\endgroup
	\fi
}

\newcommand{\ds@record@prefixfront}{
	\ifds@record@useprefix
		\begingroup
		\tikzset{byte/.append style = {compact byte}}
		\ds@bytehorizontalfalse
		\dsrecordset{hide field frame}
		\ds@record@drawbyte{}
		\ds@record@drawbyte{}
		\endgroup
		\xdef\ds@firstfieldbyte{\ds@record@chainname-byte-\thedsrbytenumber}
	\fi
}

% ds@record@prefixback: code for deteled records at the end
\newcommand{\ds@record@prefixback}{
	\ifds@record@useprefix
		\ifds@record@usedeletion
			% ignore
		\else
			\begingroup
			\tikzset{byte/.append style = {compact byte}}
			\ds@bytehorizontalfalse
			\dshexset{size}[big]{\thedsrbytenumber}
			\begin{scope}[on background layer]
				\node[byte, control bytes]
					at (\ds@record@chainname-byte-0.north west)
					{{\dsbyte{\dshexbyte{size}{0}}}};
				\node[byte, control bytes]
					at (\ds@record@chainname-byte-1.north west)
					{{\dsbyte{\dshexbyte{size}{1}}}};
				\node [control bytes block, 
					fit = {
						(\ds@record@chainname-byte-0)
						(\ds@record@chainname-byte-1)
					},
					inner sep = 0pt] (\ds@record@chainname-prefix) {};
			\end{scope}
			\endgroup
		\fi
	\fi
}


\newcommand{\ds@record@fixedback}{
	\pgfmathsetmacro\@remain{%
		int(\ds@record@recordbytelength - \thedsrbytenumber)%
	}
	\ifnum\@remain>0\relax
		\begingroup
		\tikzset{byte/.append style = {compact byte, control bytes}}
		\dsrecordset{hide field frame}
		\ds@bytehorizontalfalse
		\edef\@fragstart{\thedsrbytenumber}
		\begin{scope}[on background layer]
			\dsfieldfixed*{\@remain}{}[\recordfillbyte]
		\end{scope}
		\pgfmathsetmacro\@fragend{int(\thedsrbytenumber - 1)}
		\node[inner sep = 0pt, 
			fit = {
				(\ds@record@chainname-byte-\@fragstart.north west)
				(\ds@record@chainname-byte-\@fragend.south east)
			}] (\ds@record@chainname-fragmentation) {};
		\endgroup
	\fi
}

\newcommand{\ds@record@terminatorback}{
	\ifds@record@useterminator
		\ifnum\thedsrdatabyte>\ds@record@skipbytes\relax
			\begingroup
			\tikzset{byte/.append style = {compact byte, control bytes}}
			\dsrecordset{hide field frame}
			\ds@bytehorizontalfalse
			\begin{scope}[on background layer]
				\ds@record@drawbyte*{\ds@record@recordterminator};
			\end{scope}
			\pgfmathparse{int(\thedsrbytenumber - 1)}
			\pgfnodealias{\ds@record@chainname-terminator}
				{\ds@record@chainname-byte-\pgfmathresult}
			\stepcounter{dsrbytenumber}
			\endgroup
		\fi
	\fi		
}

%\newcommand{\ds@record@prefixback}{
%	\ifds@record@useprefix
%		\begingroup
%		\tikzset{byte/.append style = {compact byte}}
%		\dsrecordset{hide field frame}
%		\ds@bytehorizontalfalse
%		\dshexset{size}[big]{\tikzchaincount - \thedsrfieldspaces - 2}
%		\node[byte, text = white] at (\ds@record@chainname-1.north west)
%			{{\dsbyte{\dshexbyte{size}{0}}}};
%		\node[byte, text = white] at (\ds@record@chainname-2.north west)
%			{{\dsbyte{\dshexbyte{size}{1}}}};
%		\endgroup
%	\fi
%}

% dsrecord (environment): defines a record as a set of fields
% #1: record options
% #2: tikz options
\NewDocumentEnvironment{dsrecord}{oo}{

	\begingroup
	% options 
	\IfNoValueF{#1}{\dsrecordset{#1}} % record options
	\IfNoValueF{#2}{\dstikzset{#2}} % tikz options
	
	% code
	\dstikzset{%
		start chain = {\ds@record@chainname} going {
			at = (\noexpand\tikzchainprevious.north east),
			shift = (0: 0)
		}
	}
	
	% initializations
	\xdef\@ds@firstbyte{\ds@record@chainname-1} % of the record
	\xdef\ds@firstfieldbyte{\ds@record@chainname-byte-0} % of each field
	\setcounter{dsrfieldspaces}{0}
	\setcounter{dsrfieldnumber}{0} % number of the field
	\setcounter{dsrbytenumber}{0} % number of the byte
	\setcounter{dsrdatabyte}{0} % number of the current data byte
	\xdef\ds@record@skipbytes{-1}
	
	% check switches
	\ds@record@deletedfront
	\ds@record@prefixfront
}{
	\ds@record@fixedback
	\ds@record@terminatorback
	\ds@record@prefixback
	\ds@record@deletedback
	\begin{scope}[on background layer]
		\ifds@record@recordframe
			\tikzset{frame/.style = {draw, inner sep = 0.1em}}
		\else
			\tikzset{frame/.style = {inner sep = 0pt}}
		\fi
		\node[fit = {
				(\@ds@firstbyte.north west)
				(\tikzchaincurrent.south east)
			}, frame] (\ds@record@chainname) {};
	\end{scope}
	\endgroup
}


\NewDocumentEnvironment{dsrecord*}{O{}O{}}{
	\begin{dsrecord}[hide record frame, #1][#2]
}{
	\end{dsrecord}
}

\NewDocumentCommand{\dsfieldspace}{s}{
	\node[field space] {};
	\pgfmathsetmacro\@ds@last{int(\tikzchaincount + 1)}
	\xdef\ds@firstfieldbyte{\ds@record@chainname-\@ds@last}
	\IfBooleanF{#1}{\addtocounter{dsrfieldspaces}{1}}
}

\NewDocumentCommand{\dsfield}{sm}{%
	% #1: star (*)?
	% #2: contents
	\begingroup
	\IfBooleanT{#1}{
		\tikzset{byte/.append style = {compact byte}}
		\ds@bytehorizontalfalse
	}
%	\tikzset{byte/.append style = {draw}}
	\ds@record@drawfield{#2}
	\endgroup
	\xdef\ds@firstfieldbyte{\ds@record@chainname-byte-\thedsrbytenumber}
}

\NewDocumentCommand{\dsfieldterminator}{smO{\ds@record@fieldterminator}}{%
	% #1: star (*)?
	% #2: contents
	% #3: terminator
	\begingroup
	\IfBooleanT{#1}{
		\tikzset{byte/.append style = {compact byte}}
		\ds@bytehorizontalfalse
	}
	\dsrecordset{hide field frame} % draw frame only at the end 
	\ds@record@drawfield*{#2}
	\tikzset{byte/.append style = {control bytes}}
	\begin{scope}[on background layer]
		\ds@record@drawfield*{#3}
	\end{scope}
	\endgroup
	\ds@record@drawfield{} % empty, only to draw field frame
	\xdef\ds@firstfieldbyte{\ds@record@chainname-byte-\thedsrbytenumber}
}

\NewDocumentCommand{\dsfieldprefixed}{smo}{%
	% #1: star (*)?
	% #2: contents
	% #3: prefix
	\begingroup
	\IfBooleanT{#1}{
		\tikzset{byte/.append style = {compact byte}}
		\ds@bytehorizontalfalse
	}
	\IfNoValueTF{#3}{
		\dshexset{size}[big]{\ds@record@bytecount{#2}}[2]
		\begin{scope}[on background layer]
		\tikzset{byte/.append style = {control bytes}}
		\dsrecordset{hide field frame} % draw frame only at the end 
		\ds@record@drawfield*{%
			{\dsbyte{\dshexbyte{size}{0}}}%
			{\dsbyte{\dshexbyte{size}{1}}}%
		}
		\end{scope}
		\ds@record@drawfield{#2}
	}{
		\ds@record@drawfield{#3#2}
	}
	\endgroup
	\xdef\ds@firstfieldbyte{\ds@record@chainname-byte-\thedsrbytenumber}
}

\newcounter{dsrlocalbytes}
\NewDocumentCommand{\dsfieldfixed}{smmo}{%
	% #1: star (*)?
	% #2: length
	% #3: contents
	% #4: filling byte
	\begingroup
	\IfBooleanT{#1}{
		\tikzset{byte/.append style = {compact byte}}
		\ds@bytehorizontalfalse
	}
	\IfNoValueTF{#4}{
		\def\@fill{\ds@record@fieldfill}
	}{
		\def\@fill{#4}
	}
	\setcounter{dsrlocalbytes}{\ds@record@bytecount{#3}}
	\dsrecordset{hide field frame} % draw frame only at the end 
	\ds@record@drawfield*{#3}
	\def\@text{}
	\@whilenum#2>\value{dsrlocalbytes}\do{
		\g@addto@macro\@text{\@fill}
		\stepcounter{dsrlocalbytes}
	}
	\tikzset{byte/.append style = {control bytes}}
	\expandafter\ds@record@drawfield\expandafter*\expandafter{\@text}
	\endgroup
	\ds@record@drawfield{} % empty to draw the field frame
	\xdef\ds@firstfieldbyte{\ds@record@chainname-byte-\thedsrbytenumber}
}

% dsrecordstructure: draws a record with with field names and
% byte offsets
%	#1: record-related pgfkeys (opt)
%	#2: comma-separated list of field names and sizes in the form field/size
%
% todo: handle the field labels of close short-length fields close toghether
\newlength{\ds@record@fieldtextlength}
\newcommand{\dsrecordstructure}[2][]{%
	\begingroup % for options
	\pgfkeys{
		/dsdraw/record/.cd,
		#1
	}
	\dstikzset{%
		start chain = {\ds@record@chainname} going right,
		node distance = 0pt,
	}
	\setcounter{dsrrecordoffset}{\ds@record@firstbyteoffset}
	\def\recordlist{#2}
	\foreach \rlabel/\rlen/\blen [count = \n from 0] in \recordlist{
		\edef\@currentnode{\ds@record@chainname-field-\n}
		\pgfmathsetmacro\@diff{int(\rlen - \blen)}		
		\pgfmathparse{\ds@record@lengthscale * \ds@record@bytelength %
			* \blen + 1ex}
		\edef\ds@len{\pgfmathresult pt}
		\ifnum0<\@diff\relax
			% field with a break
			\def\@xdist{0.15ex}
			\def\@ydist{0.5ex}
			\node[
				record, 
				text width = \ds@len,
				draw = none,
			] (\@currentnode) {};
			\pgfnodealias{\ds@record@chainname-\rlabel}{\@currentnode}
			
			% frame
			\draw (\@currentnode.south west) -- (\@currentnode.north west);
			\draw (\@currentnode.south east) -- (\@currentnode.north east);
			\draw
				(\@currentnode.north west)
				-- ($(\@currentnode.north) - (2 * \@xdist, 0)$)
				
				($(\@currentnode.north) + (-4 * \@xdist, -\@ydist)$)
				-- ($(\@currentnode.north) + (0, \@ydist)$)

				($(\@currentnode.north) + (0, -\@ydist)$)
				-- ($(\@currentnode.north) + (4 * \@xdist, \@ydist)$)
				
				($(\@currentnode.north) + (2 * \@xdist, 0)$)
				-- (\@currentnode.north east);
			\draw
				(\@currentnode.south west)
				-- ($(\@currentnode.south) - (2 * \@xdist, 0)$)
				
				($(\@currentnode.south) + (-4 * \@xdist, -\@ydist)$)
				-- ($(\@currentnode.south) + (0, \@ydist)$)

				($(\@currentnode.south) + (0, -\@ydist)$)
				-- ($(\@currentnode.south) + (4 *\@xdist, \@ydist)$)
				
				($(\@currentnode.south) + (2 * \@xdist, 0)$)
				-- (\@currentnode.south east);
		\else
			% solid field
			\node[
				record,
				text width = \ds@len,
			] (\@currentnode) {};
			\pgfnodealias{\ds@record@chainname-\rlabel}{\@currentnode}
		\fi
		\settowidth{\ds@record@fieldtextlength}{%
			\pgfinterruptpicture\ds@record@fieldfont%
			\rlabel\endpgfinterruptpicture%
		}
		\ifdim\ds@record@fieldtextlength>\ds@len\relax
			\node[record text, yshift = 2em]
				at (\@currentnode) (label) {\rlabel\strut};
			\draw[latex-] (\@currentnode.center)
				-- ($(label.south) + (0, 0.2em)$);
		\else
			\node[record text] at (\@currentnode) {\rlabel\strut};
		\fi
		
		% render offset
		\ifds@record@showstructureoffset
			\ifnum\rlen=1\relax
				\node[record offset, anchor = north]
					at (\@currentnode.south) {\thedsrrecordoffset};
			\else
				\edef\@first{\thedsrrecordoffset}
				\addtocounter{dsrrecordoffset}{-1}
				\addtocounter{dsrrecordoffset}{\rlen}
				\edef\@second{\thedsrrecordoffset}
				\ifnum\rlen=2\relax
					\node[record offset, anchor = north]
							at (\@currentnode.south) {\@first$-$\@second};
				\else
					\node[record offset, anchor = north west]
						at (\@currentnode.south west) {\@first};
					\node[record offset, anchor = north east]
						at (\@currentnode.south east) {\@second};
				\fi
			\fi
		\fi 
		
		\pgfnodealias{last field}{\@currentnode}
		\addtocounter{dsrrecordoffset}{1}
	}
	\node[inner sep = 0pt,
		fit = {
			(\ds@record@chainname-field-0.north west)
			(last field.south east)
		}] (\ds@record@chainname) {};
	\endgroup % options
}

\ExplSyntaxOn

% dsnewnamedrecord: defines a record from a field list
%	#1: name of the record
%	#2: comma separated list of \dsfield.. especifications
%
%	Example:
%	\dssetnewrecord{myrecord}{\dsfieldprefixed, \dsfieldfixed{20}}
\NewDocumentCommand {\dsnewnamedrecord} {mm} {
	\seq_clear_new:c {g_field_spec_#1_seq}
	\seq_set_from_clist:cn {g_field_spec_#1_seq} {#2}	
}

% dsnamedrecord: produces a record from a list os contents, according to
%	a named field type specification (see \dsnewnamedrecord)
%	
%	#1: star: starred version hides record frame
%	#2: name of the record
%	#3: comma-separeted list of fields contents
\NewExpandableDocumentCommand {\dsnamedrecord} {somm} {
	\group_begin:
	\newcommand{\dsrecordputnewrecord}[2]{##1{##2}}
	\seq_clear_new:N \l_fields_args_seq
	\seq_set_from_clist:Nn \l_fields_args_seq {#4}
%	\IfNoValueF{#2}{\dsrecordset{#2}}
	\IfBooleanTF{#1}{
		\begin{dsrecord*}[#2]
	}{
		\begin{dsrecord}[#2]
	}
	\seq_mapthread_function:cNN
		 {g_field_spec_#3_seq} \l_fields_args_seq \dsrecordputnewrecord
	\IfBooleanTF{#1}{
		\end{dsrecord*}
	}{
		\end{dsrecord}
	}
	\group_end:
}
\ExplSyntaxOff

%% EOF
